\documentclass[a4paper,12pt]{report} %размер бумаги устанавливаем А4, шрифт 14пунктов
\usepackage[T2A]{fontenc}
\usepackage[utf8]{inputenc}%включаем свою кодировку: koi8-r или utf8 в UNIX, cp1251 в Windows
\usepackage[english,russian]{babel}%используем русский и английский языки с переносами
\usepackage{amssymb,amsfonts,amsmath,cite,enumerate,float,caption} %подключаем нужные пакеты расширений
\usepackage[pdftex]{graphicx} %хотим вставлять в диплом рисунки?
\graphicspath{{images/}{images/Meas POR_NOR/}}%путь к рисункам}
\usepackage[labelsep=period]{caption} %точка в подписях
\usepackage[
  locale = DE % comma as decimal mark
]{siunitx}
\makeatletter
\renewcommand{\@biblabel}[1]{#1.} % Заменяем библиографию с квадратных скобок на точку:
\makeatother

\usepackage{multirow} %объединение строк таблиц
\usepackage{multicol} %объединение столбцов таблиц
\usepackage{color} %цвет таблиц
\usepackage{colortbl} %цвет таблиц
\usepackage{bigstrut} %для таблиц
\usepackage{rotating}


\usepackage[strict]{changepage} %для смены границ полей

\usepackage{geometry} % Меняем поля страницы
\geometry{left=2cm}% левое поле
\geometry{right=1.5cm}% правое поле
\geometry{top=1cm}% верхнее поле
\geometry{bottom=2cm}% нижнее поле

\setlength{\parindent}{1cm} % настройка красной строки

\captionsetup[table]{singlelinecheck=false,justification=raggedleft} %выравнивание по правому краю заголовков таблиц

\renewcommand{\theenumi}{\arabic{enumi}}% Меняем везде перечисления на цифра.цифра
\renewcommand{\labelenumi}{\arabic{enumi}}% Меняем везде перечисления на цифра.цифра
\renewcommand{\theenumii}{.\arabic{enumii}}% Меняем везде перечисления на цифра.цифра
\renewcommand{\labelenumii}{\arabic{enumi}.\arabic{enumii}.}% Меняем везде перечисления на цифра.цифра
\renewcommand{\theenumiii}{.\arabic{enumiii}}% Меняем везде перечисления на цифра.цифра
\renewcommand{\labelenumiii}{\arabic{enumi}.\arabic{enumii}.\arabic{enumiii}.}% Меняем везде перечисления на цифра.цифра
\renewcommand{\thefigure}{\thesection.\arabic{figure}} % меням вид нумерации рисунков 
\renewcommand{\thetable}{\thesection.\arabic{table}} % меням вид нумерации таблиц
\renewcommand{\labelenumi}{\arabic{enumi}.} % меняем вид нумерованных списков
%\renewcommand{\equation}{\arabic{enumi}.} % меняем вид нумерации формул

\begin{document}
\begin{titlepage}
    \newpage
    
    \begin{center}
    ПРАВИТЕЛЬСТВО РОССИЙСКОЙ ФЕДЕРАЦИИ \\
    \vspace{1em}
    ФЕДЕРАЛЬНОЕ  ГОСУДАРСТВЕННОЕ АВТОНОМНОЕ \\
    ОБРАЗОВАТЕЛЬНОЕ УЧРЕЖДЕНИЕ ВЫСШЕГО ОБРАЗОВАНИЯ \\
    <<НАЦИОНАЛЬНЫЙ ИССЛЕДОВАТЕЛЬСКИЙ УНИВЕРСИТЕТ \\
    <<ВЫСШАЯ ШКОЛА ЭКОНОМИК>> \\
    \vspace{2em}
    \textbf{Московский институт электроники и математики им. А.Н. Тихонова}\\
    \vspace{6em}
    Степушин Кирилл Алексеевич\\
    \vspace{3em}
    \textbf{РАЗРАБОТКА ОТЛАДЧИКА С МОНИТОРИНГОМ ЭНЕРГОПОТРЕБЛЕНИЯ}\\
    \vspace{6em}
    Выпускная квалификационная работа -- магистерская диссертация\\ 
    по направлению 11.04.02 «Инфокоммуникационные технологии и системы связи»\\
    студента образовательной программы магистратуры\\
    «Интернет вещей и киберфизические системы»
    \end{center}

    \vspace{6em}

    \begin{flushleft}
    Студент \hfill Научный руководитель\\
    \hfill приглашенный преподаватель\\
    \vspace{1em}
    \rule{5cm}{0.005cm} \hfill \rule{5cm}{0.01cm}\\
    \hfill И.О.Фамилия

    \vspace{1em}

    Рецензент \hfill Консультант\\
    к.т.н., доцент \hfill приглашенный преподаватель\\
    \vspace{1em}
    \rule{5cm}{0.005cm} \hfill \rule{5cm}{0.01cm}\\
    И.О.Фамилия \hfill И.О.Фамилия
    \end{flushleft}
    
    \vspace{\fill}
    
    \begin{center}
    Москва 2024
    \end{center}
    
    \end{titlepage}%титульный лист
\tableofcontents %оглавление, которое генерируется автоматически

\chapter{Введение}
\hspace{1cm} Неазвисимо от стараний разработчика или сложности проекта, большая часть времени разработки
будет потрачена на то, чтобы убедиться, что устройство работает правильно, или -- что наиболее
вероятно -- разобраться, почему устройство работает не так, как ожидалось. Отладчик -- самый мощный 
инструмент в наборе инструментов разработчика, позволяющий напрямую взаимодействовать с процессором,
задавать точки останова, пошагово управлять потоком выполнения инструкций и проверять  значения
регистров. \cite{Lakamera:embed}

Для устройств <<интернета вещей>> очень важно знать и отслеживать энергопотребление,
ведь обычно такие устройства питаются от батарейки и каждое ненужное действие уменьшит
срок службы. Мониторинг энергопотребления позволяет понять энргоэффективность каждого сенса связи,
что позволит выбрать наиболее подходящий интерфейс и протокол передачи данных.

Об актуальности возможности мониторинга энерегопотребления для отладчика говорит количество 
измерительных устройств на рынке. Характеристики основных из них приведены в таблице 
\ref{comparemeasdevices}.

\begin{table}[H]
    \caption{Сравнение характеристик измерительных устройств}
    \label{comparemeasdevices}   
    \begin{center}
    \begin{tabular}{|c|c|c|c|c|}
    \hline
  Устройство & Joulescope & Otii Arc & NanoRanger & Current Ranger \\ \hline
    Диапазон тока & от -1 А до 3 А & от 0 до 2,5 А & от 1 нА до 800 мА & от -1,65 А до 3 А \\ \hline
    Разрешение & 1 нА & десятки нА & до 10 пА & до 1 пА  \\ \hline
    Погрешность & до 0,3\% & до 0,1\% & до 0,3\% & до 0,1\% \\ \hline
    Цена & 800 \$ & 700 \$ & 220 \$ & 120 \$  \\ \hline
    \end{tabular}
    \end{center}
\end{table} 

Так же о высокой потребности в устройстве говорит большое количество существующих отладчиков с 
мониторингом энергопотребления от различных производителей микроконтроллеров, например STLINK-V3 
\cite{STLINKV3} и Power Profiler Kit II \cite{Power Profiler Kit}, так и от сторонних компаний, например 
Energymon.

Потребность в таком отладчике также имеется у MIEM IoT-LAB для реализации возможности удаленного 
мониторинга энергопотребления, как как это сделано в оригинале у FIT IoT-LAB, а так же улучшение 
французского решения в сторону замены аппаратной реализации с PaspBerry на микроконтроллер и повышения 
точности и качества мониторинга потребляемой мощности у отлаживаемых устройств. \cite{FITIoT}.

Опеределя требования к диапазонам измеряемого тока, стоит учитывать различные IoT-утсройства. Грубую оценку
можно составить на примере Wi-Fi решений и сотовых модемов, которые в <<пике>> передачи данных могут 
иметь потребление в районе одного ампера, а в спящем режиме потребляют порядка единиц мкА.

%ссылку на инфу

Перед проектированием отладчика с возможностью мониторинга энергопотребления IoT-устройств
следует определиться с требованиями, предъявлемыми к отладчику. 
Для этого в качестве примера рассмотрим <<усредненный>> паттерн поведения устройства с BLE, одной
из самых популярных технологий беспроводной передачи данных интернета вещей, 
у которого с периодичностью в несколько десятков мс повторяется такой цикл: спящий режим 
с токопотреблением единицы мкА, далее устройство просыпается, в этот момент
энерго потребление составляет единицы мА, время просыпания -- десятки мкс, далее происходит
сеанс связи, который начинается с передачи, с токопотреблением примерно десятки мА 
и длительностю передачи <<пустого>> пакета величиной 27 байт около 200 мкс, и продолжается 
ожиданием ответа длительностью в среднем 150 мкс, после сеанс связи завершается приемом,
при котором токопотребление составляет единицы-десятки мА длительностью 200 мкс.

Зная ориенторовочные диапазоны измеряемых токов, можно грубо промоделировать данный паттерн 
поведения BLE-устройства в LTSpice, чтобы оценить необходимую полосу пропускания отладчика. 
Моделируемая схема представлена на рсиунке \ref{ris:LTSpiceScheme}, результаты моделирования 
представлены на рисунках \ref{ris:LTSpice0_01} - \ref{ris:LTSpice100} и имеют больше 
демонстрационно-оценочный характер.

\begin{figure}[H]
  \centering
  \includegraphics[scale = 0.5]{LPSpice scheme.png}
  \caption{Моделируемая схема}
  \label{ris:LTSpiceScheme}
\end{figure}

\begin{figure}[H]
  \centering
  \includegraphics[scale = 0.3]{LTSpice0_01.png}
  \caption{Результаты моделирования амперного диапазона }
  \label{ris:LTSpice0_01}
\end{figure}

Резисторы R1 -- R3 моделируют сопротивления проводников на печатной плате, конденсаторы C1 -- C2 фильтрующие 
по питанию, I1 -- источник тока, моделирующий вышеописанный паттерн поведения BLE-устройства,
R4 - шунт, для амперного диапазона, который равен 0,01 Ом (в дальнейшем в ходе дипломной работы уточняется). 
Красная линия -- входной сигнал, моделирующий потребление отлаживаемого устройства, 
зеленная -- падение напряжения на измеряемом шунте. 

\begin{figure}[H]
  \centering
  \includegraphics[scale = 0.3]{LTSpice_1.png}
  \caption{Результаты моделирования милиамперного диапазона }
  \label{ris:LTSpice_1}
\end{figure}

Здесь шунт R4 равен 1 Ом, диапазон -- милиамперный. 

\begin{figure}[H]
  \centering
  \includegraphics[scale = 0.3]{LTspice100.png}
  \caption{Результаты моделирования микроамперного диапазона }
  \label{ris:LTSpice100}
\end{figure}

А вот моделирование микроамперного диапазона, который можно считать основным измерительным диапазоном для 
IoT-устройств с малым энергопотреблением, показывает, что из-за получившегося из R1 -- R3 и C1 -- C2 RC-фильтра,
уменьшение полосы пропускания на порядки до, примерно, значения в  15 кГц, что определяет требование 
к полосе пропускания.

Так как разрабатываемое устройство является отладчиком, то для <<общения>> с отлаживаемым устройством 
в отладчике должены быть реализованы стандартные для этих целей интерфейсы, такие как UART и отладочные 
SWD/JTAG, что подразумеват под собой наличие удобных, распространеных разъемов. Так же из-за планируемого 
использования в MIEM IoT-LAB, отладчик должен уметь общаться с <<сервером>> по Ethernet, что, по сути,
является требованием заказчика, обусловленное потребностью в возможности гибкого размещения лаборатории на 
территории МИЭМа.  

Для обеспечения конкурентноспособности отладчика, остальные характиристики можно определить 
из таблицы \ref{comparemeasdevices}, а так же из анализа типичной используемой элементной базы.

Резюмируя вышесказанное, можно ориентироваться на следующие требования к разрабатываемому 
устройству:
\begin{itemize}
    \item полоса пропускания -- 15 кГц
    \item напряжение питания отлаживаемых устройств -- от 1,8 В до 12 В
    \item погрешность измерения -- до 5\%
    \item диапазон тока -- от 3,2 мА до 2 А
    \item время переключения диапазонов - десятки мкс
    \item Поддержка Ethernet, UART, SWD/JTAG
    \item себестоимость устройства -- 5000 руб.
\end{itemize}

Данные требования, предъявляемые на этапе началального анализа, в ходе более детальной проработки,
изучения и тестирования в дальнейшем будут уточнены в соответствии с полученными результатами.
% введение

\chapter{Описание структуры устройства}
\section{Подсистема управления}
\hspace{1cm} 

Проектирование любого устройства начинается с определения структуры, которая в дальнейшем
поможет составить его структурную схему. А главным компонентом любого устройства является
его подсистема управления.

Самые популярные подсистемы управления отладчиками базируются на микроконтроллерах,
которые поддерживает основные отладочные интерфейсы -- JTAG и SWD.
В качестве типичного <<отладочного>> микроконтроллера было решено использовать
STM32F107VCT6 из-за его следующих преимуществ:

\begin{itemize}
    \item \textit{Хорошо проработанная документация} -- компания
     STMicroelectronics является одним из лидеров на рынке микроконтроллеров, во многом благодаря
     замечательной документации, которая позволяет создавать на базе их решений проработанные
     и, по большей части, предсказуемо работающие проекты. Важно быть увереным, что при разработке
     устройства микроконтроллер не начнет показывать <<недокументированные>> возможности и
     различные баги, и репутация компании STMicroelectronics позволяет быть в этом
     уверенным. Антипримером может служить компания Espressif, чьи многочисленные ошибки,
     выявленные после выпуска очередного микроконтроллера, иногда выливаются в довольно
     объемные errata документы.
    \item \textit{Библиотеки} -- наличие удобных и, самое главное, пригодных в использовании 
     библиотек позолит значительно ускорить время разработки. STM32F107VCT6 построена на базе
     ядра Cortex-M3, для которого написано большое количество популярных библиотек, таких
     как HAL, LL, CMSIS, libopencm3 и другие.
    \item \textit{Большое количество готовых решений} -- некоторые из функций разрабатываемой
     системы могли быть реализованы ранее индивидуальным разработчиком, 
     сообществом или предприятием. Разработку всегда стоит начинать с поиска готовых или похожих 
     решений, которые, возможно, уже были разработаны и ждут интеграции в проект. Используемое
     в STM32F107VCT6 ядро сильно повышает шансы найти что-то готовое или то, что сильно 
     ускорит и упростит разработку устройства, позволяя не писать отдельные модули с <<нуля>>.
     \cite{Lakamera:embed}
    \item \textit{Доступность} -- в <<санкционную>> эпоху доступность компонента может стать 
     решающим фактором при выборе. Благодаро своей массовости микроконтроллеры серии STM32 
     можно легко найти как у дистрибьюторов ориентированных на крупные компание, так и на тех,
     кто работает с физическими лицами, что важно в рамках студентческой дипломной работы.
\end{itemize}

\section{Подсистема питания}
\hspace{1cm} 

Невозможно представить устройство без подсистемы питания, которая является его <<сердцем>>,
обеспечивая электроэнергией все остальные подсистемы. Плохо спроектированная система питания
может стать большой проблемой, вплоть до вывода из строя отдельной подсистемы или устройства
вцелом.

В качестве питания для отладчика была выбрана связка из PoE + преобразователь, выполненный по 
технологии fly-back. 

PoE (Power over Ethernet) — это технология передачи удаленным Ethernet-устройствам по 
витой паре электропитания вместе с данными. Данная технология позволяет питать подключенные 
устройства, к которым невозможно или нежелательно проводить кабели для питания

%В измерительных приборах вопрос питания стоит особенно остро, ведь даже те помехи, которые 
%не нанесли бы обычному цифровому устройству значительного вреда, могут с легкостью испортить
%всю точность измерения -----------сильная заготовка, но в другую главу-------------
 % описание структуры устройства
\chapter{Описание принципа работы подсистемы питания}
\section{Описание схемотехнического решения}
\hspace{1cm} 


\section{Расчет элементов схемы}
\hspace{1cm} 

\section{Результаты тестирования}
\hspace{1cm}  % описание принципа работы системы питания
\chapter{Описание принципа работы подсистемы измерения энергопотребления}
\section{Описание схемотехнического решения}
\hspace{1cm} 


\section{Расчет элементов схемы}
\hspace{1cm} 

\section{Результаты тестирования}
\hspace{1cm}  % описание принципа работы измерительной части
\begin{thebibliography}{00}

\addcontentsline{toc}{chapter}{Список используемой литературы}

\bibitem{Lakamera:embed} Лакамера, Д.
\emph{Архитектура встраиваемых систем} /Д. Лакамера // ДМК Пресс --
Москва -- 2023. -- 332 с.

\end{thebibliography}
 % ссылки на литературу
\end{document}