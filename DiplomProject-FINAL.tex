\chapter*{Заключение}
\addcontentsline{toc}{chapter}{Заключение}
\hspace{1cm}

По результатам выполнения магистерской диссертации был разработан удаленный отладчик с возможностью 
мониторинга энергопотребления.

В ходе работы был проведен обзор существующих измерителей энергопотребления, а так же отладчиков с 
такой возможностью. Была сформирована и описана структурная схема устройства, особенности и 
причины выбора каждой его подсистемы. 

Также, в ходе работы был подробно разобран принцип работы каждого узла подсистемы питания, 
был проведен расчет основных элементов обвязки DC-DC преобразователя, реализована технология PoE 
в соответствии стандарту  IEEE 802.3af, спроектирована схема регулируемого преобразователя для 
питания отлаживаемых устройств. Кроме того, были приведены результаты работы реализации PoE и 
изолированного fly-buck преобразователя в виде осциллограмм. 

В магистерской диссертации были разобраны преимущества и недостатки схем измерения тока по принципу 
нижнего и верхнего плеча, рассмотрена и реализована схема двунаправленного токового измерителя, в основе 
которой лежит дифференциальный усилитель. В ходе подбора ОУ были рассмотрены принципы работы
классических решений для получения наименьшего напряжения смещения, такие как чопперная стабилизация
и DigiTrim. Были исследованы показания напряжения смещения для различных ОУ, а так же их реакция на 
положительную и отрицательную перегрузку, что позволило обоснованно подойти к выбору ключевого 
компонента измерительной части устройства. Кроме того, была спроектирована и описана схема подсистемы 
измерения энергопотребления, рассмотрены различные варианты ее реализации, что позволило выбрать 
оптимальное решение. 

В ходе работы была разработана подсистема Ethernet, отвечающая за реализацию физического уровня 
модели OSI, рассмотрен принцип дифференциальной передачи сигнала, произведен расчет целевого 
импеданса линий передачи данных по Ethernet, а так же разобрана применяемая схема PHY-контроллера. 

Кроме того, было доработано и применено программное обеспечение с открытым исходным кодом, что 
позволило реализовать весь планируемый функционал устройства. 

При дальнейшей работе стоит сделать акцент на более детальную проработку печатной платы устройства, 
возможной ее адаптации для мелкосерийного производства, оптимизации написанного программного 
обеспечения, а также на добавлении возможности поддержки большего количества отлаживаемых устройств 
<<интернета вещей>>. 

Результаты работы рекомендуется использовать при разработке отладчиков с мониторингом энергопотребления, 
а также других похожих решений Интернета вещей.