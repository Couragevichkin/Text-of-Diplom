
\chapter*{Аннотация}
\hspace{1cm} 

Особенностью устройств <<Интернета вещей>> является зависимость их энергопотребления от множества 
внешних факторов, что усложняет их отладку непосредственно на рабочем месте программиста 
встраиваемых систем. Кроме того, существует потребность в массированной удаленной отладке устройств, 
использующих различные протоколы радиосвязи. Разрабатываемое устройство предназначено для работы в 
составе лаборатории с удаленным доступом, аналогичной французскому проекту FIT IoT-LAB и реализует 
необходимую для этого функциональность, отсутствующую в имеющихся на рынке отладчиках.
В данной работе представлено описание существующих отладчиков с мониторингом энергопотребления, 
а так же описан процесс разработки такого отладчика. В ходе работы был проведен анализ существующих 
решений для каждого функционального узла, а так же предложена общая структура отладчика. В работе
приведена реализация подсистемы питания, которая состоит из PoE контроллера, изолированного fly-buck
DC-DC преобразователя и регулятора напряжения. Кроме того, в работе описана реализация физического 
уровня модели OSI, а также разработан двунаправленный измеритель тока, в основе которого 
лежит дифференциальный усилитель. Кроме того, в работе описана реализация подсистемы преобразования 
логических уровней напряжения. Результаты работы предполагается использовать в лаборатории 
MIEM IoT-LAB. Работа состоит из 63 страниц и включает 53 рисунка, 6 таблиц и 1 приложение.
Был использован 41 источник.


