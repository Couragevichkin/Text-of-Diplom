\chapter{Описание принципа работы подсистемы измерения энергопотребления}
\section{Обоснование выбора компонентов измерительной части}
\subsection{Операционный усилитель}
\hspace{1cm} 

Вопрос какой операционный усилитель (далее ОУ) использовать в качестве дифференциального усилителя -- самый 
критичный для подсистемы измерения энергопотребления. Прежде всего стоит определиться с требованиями к 
самым важным параметрам ОУ. 
Одной из важных характеристик операционного усилителя (ОУ) является напряжение смещения $V_{os}$ — или, 
говоря проще, напряжение ошибки на его входах. Любой неидеальный ОУ при отсутствии входного сигнала выдаёт 
выходной так, как будто бы на самом деле вход-ной сигнал равен Vos.

Напряжение смещения обычно составляет от единиц микровольт до единиц милливольт — и, соответственно, 
доставляет серьёзные неудобства при работе с низковольтными источни-ками: термопарами, шунтами и так далее.
Особенно — если оно отрицательное, а схема однополярная: тогда на выходе ОУ будет просто 0, пока 
напряжение входного сигнала не превысит Vos, и никакой калибровкой это устранить невозможно \cite{Chopper:OU} 
\cite{MT-037:Tutorial}. 

При измерении микро- и милиамперных диапазонов высокое напряжение смещение может стать проблемой.
Для этого следует изучить реальную зависимость напряжения смещения от Vcm, -- синфазного
напряжения на входах ОУ, и сравнить ее с той, 
которую приводят в datasheet, как, например, на рисунке \ref{ris:411} \cite{OPAx376:datasheet}.
\begin{figure}[H]
\centering
\includegraphics[scale = 0.8]{ris411.png}
\caption{Зависимость напряжения смещения от $V_{cm}$}
\label{ris:411}
\end{figure}

Так же стоит посмотреть на напряжение смещения в зависимости от ОУ в партии, 
на рисунке \ref{ris:412} представлен график для OPA2376 \cite{OPAx376:datasheet}

\begin{figure}[H]
\centering
\includegraphics[scale = 0.8]{ris412.png}
\caption{Показатель напряжения смещения в рамках одной партии}
\label{ris:412}
\end{figure}

Также стоит упомянуть про то, что при использовании автоматического переключения шунтов 
чоппер-стабилизированные ОУ малопригодны из-за долгого времени восстановления, так как из-за частоты работы
АЦП порядка сотен кГц, время переключения свыше 1 мкс нам не подходит \cite{Chopper:OU}.

Исходя из вышесказанного, можно изучить следующие ОУ:

\begin{itemize}
    \item OPA2376 от Texas Instruments -- прецизионный rail-to-rail, выполненный по технологии etrim
    \item OPA2376 от Fulihao -- в ходе дальнейших экспериментов было выяснено, что это 
    чоппер-стабилизированный ОУ
    \item AD8606 от Analog Devices -- прецизионный rail-to-rail, выполненный по технологии etrim
    \item TP2312 -- прецизионный малошумящий rail-to-rail
    \item RS8552 -- чоппер-стабилизированный
    \item RS8562 -- чоппер-стабилизированный
\end{itemize}

На выбор ОУ влияла так же возможность быстро и без проблем приобрести на территории РФ. 

На рисунке \ref{ris:413} представлена схема измерения напряжения смещения.

\begin{figure}[H]
    \centering
    \includegraphics[scale = 0.9]{ris413.png}
    \caption{Схема измерительной установки}
    \label{ris:413}
    \end{figure}

Здесь Х1 -- контактирующее устройство, предназначенное для быстрой смены операци-онного усилителя в 
корпусе SOIC-8 -- самого распространённого типа корпуса для ОУ. Испытуемый ОУ включён по схеме неинвертирующего 
усилителя с коэффициентом усиления 1000, который обеспечивается резисторами R2\_1 и R1\_1 для первого ОУ и
резисторами R2\_2 и R1\_2 для второго ОУ в корпусе. DA1 с обвязкой к нему выполняет роль буферного ОУ для 
обеспечения лучшего импеданса и большей нагрузочной способности при низком токе через делитель R3 и R4.

Данная схема позволит преобразовать микровольтное напряжение смещение в миливольты, что достаточно для 
измерения обычным вольтметром. Подстроечным резистором RP1 регулируется input common-mode voltage, как 
на рисунке \ref{ris:411}. Напряжение питание схемы 5 В. Результаты измерений предаставлены в таблице
\ref{tab:Vcm1} для OPA2376 всех производителей, AD8606 и TP2312, и в таблице \ref{tab:Vcm2} для RS8552 и 
RS8562.

% Table generated by Excel2LaTeX from sheet 'Vcm' and from my teardrops
\begin{table}[H]
    \begin{adjustwidth}{-1em}{}
    \centering
    \caption{Результаты измерений напряжения смещения у разных ОУ}
      \begin{tabular}{|c|c|c|c|c|c|c|c|c|c|c|c|}
      \hline
      \multicolumn{1}{|c|}{\multirow{3}[6]{2cm}{\textbf{Тип ОУ}}} & \multicolumn{1}{c|}{\multirow{3}[6]{1.4cm}{\textbf{Номер экземпляра}}} & \multicolumn{1}{c|}{\multirow{3}[6]{1.8cm}{\textbf{Номер ОУ в корпусе}}} & \multicolumn{9}{|c|}{\textbf{Напряжение смещения, мВ}} \bigstrut\\
  \cline{4-12}          &       &       & \multicolumn{9}{|c|}{\textbf{При Vcm, В}} \bigstrut\\
  \cline{4-12}          &       &       & \textbf{0,5} & \textbf{1} & \textbf{1,5} & \textbf{2} & \textbf{2,5} & \textbf{3} & \textbf{3,5} & \textbf{4} & \textbf{4,5} \bigstrut\\
      \hline
      \multicolumn{1}{|c|}{\multirow{10}[20]{*}{OPA2376 }} & \multirow{2}[4]{*}{1} & 1     & 13,4  & 7,9   & 2,4   & -2,4  & -8,1  & -11,8 & -15,3 & -12,8 & 483 \bigstrut\\
  \cline{3-12}          &       & 2     & 4,2   & 6     & 4,3   & 2,3   & -1    & -3,5  & -6    & -9,7  & -1813 \bigstrut\\
  \cline{2-12}          & \multirow{2}[4]{*}{2} & 1     & 28,5  & 18,7  & 9,3   & 0,3   & -8    & -15,2 & -22,7 & -58,5 & -388 \bigstrut\\
  \cline{3-12}          &       & 2     & -5,5  & -7,5  & -8,5  & -6,8  & -6,1  & -3,9  & -1,9  & -8,9  & -680 \bigstrut\\
  \cline{2-12}          & \multirow{2}[4]{*}{3} & 1     & 26,7  & 20,3  & 14,2  & 9,1   & 3,6   & -0,2  & -4,4  & -16,6 & -2230 \bigstrut\\
  \cline{3-12}          &       & 2     & -55,7 & -33,4 & -21,7 & -12,3 & -7,3  & -1,5  & 2,9   & -10,9 & -4500 \bigstrut\\
  \cline{2-12}   TI     & \multirow{2}[4]{*}{4} & 1     & -18,2 & -10,8 & -6,3  & -2,5  & 0,1   & 2,5   & 4,6   & 14,5  & -1532 \bigstrut\\
  \cline{3-12}          &       & 2     & 7,2   & 5,5   & -0,1  & -4,8  & -9,2  & -13,4 & -17,2 & -23,3 & -1517 \bigstrut\\
  \cline{2-12}          & \multirow{2}[4]{*}{5} & 1     & -18,8 & -10,4 & -4,5  & 0,2   & 3,2   & 6,3   & 8,6   & 22    & 395 \bigstrut\\
  \cline{3-12}          &       & 2     & -70,4 & -43,6 & -24,7 & -11,2 & -1,6  & 7,5   & 14,4  & 30    & 624 \bigstrut\\
      \hline
      \multicolumn{1}{|c|}{\multirow{6}[12]{*}{OPA2376}} & \multirow{2}[4]{*}{1} & 1     & -6,3  & -6,5  & -8,4  & -8,2  & -7,35 & -7,9  & -10   & -5,9  & -4,6 \bigstrut\\
  \cline{3-12}          &       & 2     & -2,8  & -2,9  & -4,5  & -4,3  & -3,1  & -3,3  & -4,7  & -2,7  & -1,7 \bigstrut\\
  \cline{2-12}          & \multirow{2}[4]{*}{2} & 1     & -6,8  & -6,8  & -9    & -9    & -8,01 & -8,9  & -10,4 & -11,8 & -10,1 \bigstrut\\
  \cline{3-12}          &       & 2     & 1,3   & 1     & -0,6  & -0,5  & 0,2   & 1,2   & -0,1  & 2     & 3,4 \bigstrut\\
  \cline{2-12} Fulihao   & \multirow{2}[4]{*}{3} & 1     & -4,6  & -4,4  & -6,4  & -6,8  & -6,4  & -6,5  & -8,5  & -7,7  & -6,2 \bigstrut\\
  \cline{3-12}          &       & 2     & -1,4  & -1,2  & -2,3  & -2,1  & -0,8  & -0,6  & -2,3  & -1,5  & 0,2 \bigstrut\\
      \hline
      \multicolumn{1}{|c|}{\multirow{10}[20]{*}{AD8606}} & \multirow{2}[4]{*}{1} & 1     & 40,1  & 36,3  & 33,7  & 33,5  & 31,3  & 30,6  & 3,5   & 6,2   & 19,8 \bigstrut\\
  \cline{3-12}          &       & 2     & 11,4  & 32,8  & 49,3  & 59,67 & 72    & 80,5  & 19,2  & -10,4 & -13,6 \bigstrut\\
  \cline{2-12}          & \multirow{2}[4]{*}{2} & 1     & 21,4  & 27,2  & 31,3  & 34,5  & 36,1  & 38,3  & 87,5  & 73,8  & 93,6 \bigstrut\\
  \cline{3-12}          &       & 2     & 12,8  & 4,6   & 0,6   & -0,3  & -2,2  & -2,1  & -94,4 & -39,9 & -46,5 \bigstrut\\
  \cline{2-12}          & \multirow{2}[4]{*}{3} & 1     & 17,8  & 34,3  & 47,5  & 59,2  & 71,8  & 84,9  & 29,7  & 29,5  & 23,6 \bigstrut\\
  \cline{3-12}          &       & 2     & -16,3 & -3,3  & 6,6   & 13,9  & 1,3   & 27,9  & 48,8  & 9,3   & 15,6 \bigstrut\\
  \cline{2-12}          & \multirow{2}[4]{*}{4} & 1     & 74,7  & 63,5  & 51,1  & 40,5  & 31,9  & 24,1  & 40,9  & 61,6  & 73,7 \bigstrut\\
  \cline{3-12}          &       & 2     & 15,3  & -6,9  & -25,1 & -38,5 & -51,4 & -64,2 & -53,8 & -17,5 & -29,4 \bigstrut\\
  \cline{2-12}          & \multirow{2}[4]{*}{5} & 1     & 59,5  & 51,8  & 41,9  & 35,3  & 24,8  & 18,3  & 10,4  & 0,9   & 11,8 \bigstrut\\
  \cline{3-12}          &       & 2     & 14,9  & 22,5  & 32,1  & 37,4  & 44,2  & 48,6  & 35,2  & 35,7  & 40,4 \bigstrut\\
      \hline
      \multicolumn{1}{|c|}{\multirow{10}[20]{*}{TP2312}} & \multirow{2}[4]{*}{1} & 1     & 17,4  & 18,2  & 18,8  & 20,6  & 23,3  & 26,5  & 28,9  & 165,9 & -1072 \bigstrut\\
  \cline{3-12}          &       & 2     & 30,4  & 38,6  & 43,6  & 47,9  & 49,7  & 50,6  & 51,2  & -41,7 & -1550 \bigstrut\\
  \cline{2-12}          & \multirow{2}[4]{*}{2} & 1     & -12,4 & -10,2 & -7    & -2,6  & 3,2   & 7     & 12,9  & -391  & -3330 \bigstrut\\
  \cline{3-12}          &       & 2     & 8,3   & 14,9  & 19,3  & 22,4  & 26,9  & 30,8  & 34,2  & 24,3  & -1183 \bigstrut\\
  \cline{2-12}          & \multirow{2}[4]{*}{3} & 1     & 12,8  & 6,6   & 2,3   & -4,2  & -10,9 & -16,9 & 46,3  & 34,9  & -403 \bigstrut\\
  \cline{3-12}          &       & 2     & 16,2  & 15,4  & 17,1  & 19,4  & 23    & 27,5  & 51,2  & 330   & -1491 \bigstrut\\
  \cline{2-12}          & \multirow{2}[4]{*}{4} & 1     & -24,1 & -16,9 & -11   & -4,3  & 0,5   & 4,1   & 8,4   & 269   & -2530 \bigstrut\\
  \cline{3-12}          &       & 2     & 3,7   & 3,9   & 4,5   & 6,8   & 9,1   & 12,9  & 16,7  & 468   & -1950 \bigstrut\\
  \cline{2-12}          & \multirow{2}[4]{*}{5} & 1     & 3,6   & -13,8 & -23,8 & -31,8 & -38,5 & -43,8 & -47,3 & -123,2 & -2660 \bigstrut\\
  \cline{3-12}          &       & 2     & 28,2  & 23    & 18,5  & 14    & 8,8   & 5,3   & 1,3   & 934   & -171 \bigstrut\\
      \hline
      \end{tabular}%
    \label{tab:Vcm1}%
    \end{adjustwidth}
  \end{table}



  \begin{table}[H]
    %\begin{adjustwidth}{-2em}{}
    \centering
    \caption{Результаты измерений напряжения смещения у разных ОУ}
      \begin{tabular}{|c|c|c|c|c|c|c|c|c|c|c|c|}
      \hline
      \multicolumn{1}{|c|}{\multirow{3}[6]{2cm}{\textbf{Тип ОУ}}} & \multicolumn{1}{c|}{\multirow{3}[6]{1.4cm}{\textbf{Номер экземпляра}}} & \multicolumn{1}{c|}{\multirow{3}[6]{1.8cm}{\textbf{Номер ОУ в корпусе}}} & \multicolumn{9}{|c|}{\textbf{Напряжение смещения, мВ}} \bigstrut\\
  \cline{4-12}          &       &       & \multicolumn{9}{|c|}{\textbf{При Vcm, В}} \bigstrut\\
  \cline{4-12}          &       &       & \textbf{0,5} & \textbf{1} & \textbf{1,5} & \textbf{2} & \textbf{2,5} & \textbf{3} & \textbf{3,5} & \textbf{4} & \textbf{4,5} \bigstrut\\
      \hline
      \multicolumn{1}{|c|}{\multirow{10}[20]{*}{RS8552}} & \multirow{2}[4]{*}{1} & 1     & -0,1  & -0,4  & -0,5  & 0,3   & -0,6  & -0,4  & 0,9   & -0,8  & -0,5 \bigstrut\\
  \cline{3-12}          &       & 2     & -1,3  & -0,5  & -0,4  & -0,1  & -0,5  & 0,6   & -0,2  & -0,2  & -0,9 \bigstrut\\
  \cline{2-12}          & \multirow{2}[4]{*}{2} & 1     & -0,3  & -0,4  & -0,1  & -0,4  & -0,2  & -0,3  & -0,2  & -0,6  & -5 \bigstrut\\
  \cline{3-12}          &       & 2     & -0,4  & -0,2  & 0     & -0,1  & -0,8  & -0,5  & -0,3  & -0,5  & -0,5 \bigstrut\\
  \cline{2-12}          & \multirow{2}[4]{*}{3} & 1     & -0,5  & 0,5   & 0,2   & -0,6  & -0,2  & -0,9  & -0,7  & -0,3  & -0,4 \bigstrut\\
  \cline{3-12}          &       & 2     & -0,3  & -0,6  & -0,3  & -0,2  & -0,3  & -0,2  & -0,8  & -0,4  & -0,5 \bigstrut\\
  \cline{2-12}          & \multirow{2}[4]{*}{4} & 1     & 0,8   & 1,1   & 1,4   & 1     & 0,8   & -0,7  & -0,1  & -0,6  & -0,3 \bigstrut\\
  \cline{3-12}          &       & 2     & -0,6  & -0,4  & -0,6  & -0,2  & -0,4  & -0,3  & -0,1  & -0,7  & -0,9 \bigstrut\\
  \cline{2-12}          & \multirow{2}[4]{*}{5} & 1     & -0,4  & 0,1   & -0,4  & -0,1  & -0,2  & -0,2  & -0,3  & -0,5  & -0,4 \bigstrut\\
  \cline{3-12}          &       & 2     & -0,3  & -0,3  & -0,2  & -0,5  & -0,2  & -0,1  & -0,6  & -0,5  & -0,6 \bigstrut\\
      \hline
      \multicolumn{1}{|c|}{\multirow{10}[20]{*}{RS8562}} & \multirow{2}[4]{*}{1} & 1     & -1,3  & -0,6  & -0,4  & -0,3  & -0,1  & -0,9  & -0,4  & -0,6  & -0,1 \bigstrut\\
  \cline{3-12}          &       & 2     & -1,3  & -0,4  & -0,9  & -0,5  & -0,8  & -0,3  & -0,4  & -0,8  & -0,7 \bigstrut\\
  \cline{2-12}          & \multirow{2}[4]{*}{2} & 1     & 0,1   & 0,5   & 0,1   & -0,6  & 0,5   & -0,2  & -0,2  & -0,2  & 0,3 \bigstrut\\
  \cline{3-12}          &       & 2     & -1,2  & -0,9  & -1    & -0,5  & -0,2  & -0,9  & -0,2  & -0,5  & -0,8 \bigstrut\\
  \cline{2-12}          & \multirow{2}[4]{*}{3} & 1     & 0     & 0,3   & 0,2   & -0,2  & 0,6   & -0,8  & -0,7  & -0,6  & 0,4 \bigstrut\\
  \cline{3-12}          &       & 2     & -0,8  & -0,1  & -1,2  & -0,1  & -0,1  & -0,4  & -0,6  & -0,6  & -0,9 \bigstrut\\
  \cline{2-12}          & \multirow{2}[4]{*}{4} & 1     & 0,2   & 0,6   & -0,3  & -0,7  & -0,2  & -0,9  & -0,7  & -0,2  & -0,1 \bigstrut\\
  \cline{3-12}          &       & 2     & -1,8  & -1,5  & -2,1  & -0,6  & -0,6  & -0,1  & -0,7  & -0,4  & -0,8 \bigstrut\\
  \cline{2-12}          & \multirow{2}[4]{*}{5} & 1     & -0,1  & 0,2   & 0,4   & -0,9  & 0,3   & -0,6  & -0,8  & 0,1   & -0,6 \bigstrut\\
  \cline{3-12}          &       & 2     & -1,6  & -1    & -0,6  & -0,4  & -0,5  & -0,9  & -0,1  & -0,9  & -0,7 \bigstrut\\
      \hline
      \end{tabular}%
    \label{tab:Vcm2}%
    %\end{adjustwidth}
  \end{table}
%\end{landscape}
В данных таблицах результаты измерения после усиления напряжения смещения в 1000 раз. Из результатов измерений
можно сказать, что заявленное производителями напряжение смещения соответствует измеренному. 

Для RS8552 и RS8562 падение напряжения сильно менялось в ходе измерений, в таблице представлено среднее
значение, что может говорить о внешних наводках и помехах, что вряд ли, так как такое поведение было 
замечено только у этих двух ОУ, либо о том, что эти операционные усилители ведут себя нестабильно при малом 
выходном напряжении.

Так же важным параметром, влияющим на быстродействие, что важно для динамической нагрузки, является 
время воостановления после перегрузки. Под перегрузкой подразумевается  любое состояние, при котором 
сигнал на выходе операционного усилителя достигает крайних значений как максимального, так и минимального,
так как на этих граничных состояниях нарушается правило одинаковости сигналов на входах ОУ \cite{Chopper:OU}.

Для измерения времен восстановления используем схему, изображенную на рисунке \ref{ris:414}.

\begin{figure}[H]
\centering
\includegraphics[scale = 0.7]{ris414.png}
\caption{Схема установки для измерения времен восстановления после перегрузки}
\label{ris:414}
\end{figure}

На этой схеме первый ОУ в корпусе DA21A задает стабильные 2,5 В, полученные с делителя R21, R22 с 
коэффициентом деления равным 2, на неинвертирующем входе второго ОУ в корпусе DA21B, включенного по схеме
инвертирующего усилителя с коэффициентом усиления равным -100. 


Осцилограммы измерения время восстановления после положительной перегрузки (Positive Over-Votage Recovery) 
и после отрицательной перегрузки (Negative Over-Votage Recovery) для вышеперечисленных операционных усилитей
предаставлены на рисунках \ref{ris:415} - \ref{ris:426}. На всех рисунках желтый сигнал - на инвертирующем 
входе ОУ, а розовый - сигнал с выхода ОУ. Развертка по вертикали у входного сигнала -- 50 мВ/деление, 
у выходного 1 В/деление. 

\begin{figure}[H]
\centering
\includegraphics[scale = 0.4]{AD8606_NOR.jpg}
\caption{Negative Over-Votage Recovery у AD8606}
\label{ris:415}
\end{figure}

\begin{figure}[H]
\centering
\includegraphics[scale = 0.4]{AD8606_POR.jpg}
\caption{Positive Over-Votage Recovery у AD8606}
\label{ris:416}
\end{figure}

У AD8606 Negative Over-Votage Recovery (здесь и далее NOR) примерно равно 5 мкс, развертка по времени 
2 мкс/деление, Positive Over-Votage Recovery (здесь и далее POR) примерно равно 450 нс, развертка по времени 
200 нс/деление.

\begin{figure}[H]
\centering
\includegraphics[scale = 0.4]{OPA2376 Fulihao_NOR.jpg}
\caption{Negative Over-Votage Recovery у OPA2376 Fulihao}
\label{ris:417}
\end{figure}

\begin{figure}[H]
\centering
\includegraphics[scale = 0.4]{OPA2376 Fulihao_POR.jpg}
\caption{Positive Over-Votage Recovery у OPA2376 Fulihao}
\label{ris:418}
\end{figure}

У OPA2376 Fulihao NOR примерно равно 90 мкс, развертка по времени 50 мкс/деление, 
POR примерно равно 600 нс, развертка по времени 200 нс/деление.

\begin{figure}[H]
\centering
\includegraphics[scale = 0.4]{OPA2376 TI_NOR.jpg}
\caption{Negative Over-Votage Recovery у OPA2376 TI}
\label{ris:419}
\end{figure}

\begin{figure}[H]
\centering
\includegraphics[scale = 0.4]{OPA2376 TI_POR.jpg}
\caption{Positive Over-Votage Recovery у OPA2376 TI}
\label{ris:420}
\end{figure}

У OPA2376 TI NOR примерно равно 4,25 мкс, развертка по времени 2 мкс/деление, 
POR примерно равно 600 нс, развертка по времени 500 нс/деление. 

Здесь видна разительная разница во временах восстановления после отрицательной перегрузки у OPA2376 от 
разных производителей. Разница составляет больше, чем в 20 раз. Это свидетельсвует о том, что OPA2376 от 
Fulihao на самом деле не etrim, а чоппер-стабилизированный, ведь именно для этого типа ОУ характерны такие 
времена восстановления.
Это не только делает OPA2376 от Fulihao непригодным к использованию для наших целей, но и заставляет 
внимательнее следить в целом за всеми заказанными OPA2376, особенно от китайских производителей.

\begin{figure}[H]
\centering
\includegraphics[scale = 0.4]{RS8552_NOR.jpg}
\caption{Negative Over-Votage Recovery у RS8552}
\label{ris:421}
\end{figure}

\begin{figure}[H]
\centering
\includegraphics[scale = 0.4]{RS8552_POR_2.jpg}
\caption{Positive Over-Votage Recovery у RS8552}
\label{ris:422}
\end{figure}

У RS8552 NOR примерно равно 60 мкс, развертка по времени 20 мкс/деление, 
POR примерно равно 750 нс, развертка по времени 1 мкс/деление.

\begin{figure}[H]
\centering
\includegraphics[scale = 0.4]{RS8562_NOR.jpg}
\caption{Negative Over-Votage Recovery у RS8562}
\label{ris:423}
\end{figure}

\begin{figure}[H]
\centering
\includegraphics[scale = 0.4]{RS8562_POR_2.jpg}
\caption{Positive Over-Votage Recovery у RS8562}
\label{ris:424}
\end{figure}

У RS8562 NOR примерно равно 22,5 мкс, развертка по времени 10 мкс/деление, 
POR примерно равно 300 нс, развертка по времени 100 нс/деление.

\begin{figure}[H]
\centering
\includegraphics[scale = 0.4]{TP2312_NOR.jpg}
\caption{Negative Over-Votage Recovery у TP2312}
\label{ris:425}
\end{figure}

\begin{figure}[H]
\centering
\includegraphics[scale = 0.4]{TP2312_POR_2.jpg}
\caption{Positive Over-Votage Recovery у TP2312}
\label{ris:426}
\end{figure}

У TP2312 NOR примерно равно 2,25 мкс, развертка по времени 2 мкс/деление, 
POR примерно равно 200 нс, развертка по времени 100 нс/деление.

По результатам всех проведенных измерений выбор основного измерительнго операционного усилителя пал на 
TPS2376, у него одно из самых малых напряжений смещений, не было замечено нестабильностей в работе, 
а так же у него оказались самые маленькие времена восстановлений после перегрузки. Так же стоит отметить 
его доступность в РФ.

\subsection{Измерительный шунт}
\hspace{1cm} 

Важным элементом в измерительной схеме выступает шунт, — его физические свойства, 
такие как сопротивление, максимальная мощность или температурный коэффициент, 
сильно влияют на точность всего измерения. Поэтому выбор подходящей модели шунтирующего резистора 
важен для коррекнтых измерений. Например, слишком высокое сопротивление шунта может привести к 
значительному падению значения выходного напряжения ниже допустимого уровня, что приведет к снижению 
эффективности устройства. Кроме того, большая мощность, рассеиваемая на шунте, повысит его температуру, 
что дестабилизирует его рабочие параметры и ухудшит точность измерений. Так же высокий температурный 
коэффициент сопротивления (ТКС) сделает всю измерительную часть нетермостабильной, что так же нарушит 
корректность измерений. Этот параметр описывает повышение значений сопротивления элемента в зависимости от
его температуры -- чем больше ТКС, тем ниже точность измерения.

Другой важной характеристикой шунтирующего резистора является тепловой коэффициент ЭДС. На стыке 
соединений двух разных металлов создается электродвижущая сила порядка микровольт. Величина этой 
силы (и создаваемое ею напряжение) изменяется в зависимости от температуры. Эти изменения описываются 
тепловым коэффициентом (чаще всего выражают в мкВ/°С). Шунтирующие резисторы могут работать в широком 
диапазоне измерений -- при измерении очень малых токов дополнительное напряжение, вызванное фактором 
ЭДС, может значительно исказить результаты.

Для того чтобы значение тока измерялось с хорошей точностью следует отказаться от упрощенной модели 
состоящей из одного значения сопротивления, заменив ее более сложной, хотя и более реалистичной моделью, 
состоящей из трех последовательно соединенных сопротивлений. Это номинальное и двухкомпонентное 
сопротивление. В случае обычных резисторов сопротивления выводов имеют пренебрежимо малые значения, 
в случае шунтирующих, характеризующихся очень малыми значениями номинального сопротивления, 
эти дополнительные паразитные параметры вносят существенный вклад в работу, а игнорирование их 
влияния приводит к увеличению погрешности измерения. Схема замещения шунта изображена на рисунке 
\ref{ris:427}.

\begin{figure}[H]
\centering
\includegraphics[scale = 1]{ris427.png}
\caption{Схема замещения шунтирующего резистора}
\label{ris:427}
\end{figure}

Для достижения требуемой точности измерения тока в 0,5\% на разных диапазонах измеряемого тока, 
требуется использовать шунты разного номинала. С учетом требований к устройсту, сформированных в главе 1,
номиналы шунтов, которых хватит для покрытия всего измеряемого диапазона, были выбраны 
0,01 Ом, 1 Ом, и 100 Ом. С точки зрения подбора ЭКБ самыми проблемными являются шунты на 0,01 Ом. 
В ходе поиска и подбора шунтов выбор пал на серию WSL2512 для шунтов 0,01 Ом, так как шунты данной серии
подходят под все требования \cite{GooglePatent:1}. 

\section{Тестирование схем измерения}
\hspace{1cm}

Даже в рамках измерения по схеме нижнего плеча существуют разные решения. Расмотрим рисунки \ref{ris:428} --
\ref{ris:429}.

\begin{figure}[H]
\centering
\includegraphics[scale = 0.58]{Meas_LM7705.png}
\caption{Схема измерения с LM7705}
\label{ris:428}
\end{figure}


\begin{figure}[H]
\centering
\includegraphics[scale = 0.63]{Meas_whithout_LM7705.png}
\caption{Схема измерения без LM7705}
\label{ris:429}
\end{figure}

В основе данных схем лежит операционный усилитель DA1B, включенный по схеме дифференциального усилителя с 
коэффициентом усиления 51, который задан резисторами R3-R6. Суть его работы заключается в усилении разницы 
напряжений между входами. На неинвертирующий вход подано смещение 2,5 В, которое задается делителем R1, R2 и 
стабилизируется ОУ DA1A, включенным по схеме повторителя. С разъема Х3 снимается выходное усиленное напряжение. 

Разъем X2 предназначен для подключения источника постоянного тока, имитирующего ток потребления отлаживаемых устройств. 
Транзисторы VT2 -- VT4, включенные по ключевой схеме задают, при падоче на их затвор высокого уровня напряжения,
через какие шунты R12, R15, R18 протечет ток с разъема Х2, создавая определенное падение напряжения. 
Очевидно, что при включении транзистора VT2 ток будет протекать через
все шунты, общим сопротивлением 101,01 Ом, создавая падение напряжения, согласно закону Ома, равное 
$U_{drop} = I_{source} \cdot R_{shunt} = 101,01 \cdot I_{source}$. При включении транзистора VT3 падение напряжения
на шунтах будет составлять $1,01 \cdot I_{source}$, и $0,01 \cdot I_{source}$ при включении транзистора VT4. 

Резисторы R10, R11, R13, R14, R16, R17 предназначены для защиты управляющего транзисторами элемента от короткого
замыкания, путем ограничения протекающего тока. 

Это работает следующим образом: у каждого МОП-транзистора, из-за наличия у него pn-перехода, имеются
паразитные емкости $C_{gate-source}$, $C_{gate-drain}$, $C_{drain-source}$. Не так важно их значение, как то, что
переходная характеристика любой емкости в начальный момент зарядки имеет импеданс близкий к нулю, что 
показано на рисунке \ref{ris:430}

\begin{figure}[H]
\centering
\includegraphics[scale = 0.55]{C_transtion.png}
\caption{Переходная характеристика идеального конденсатора}
\label{ris:430}
\end{figure}

, где из-за очень маленького начального соотношения $\frac{Y_{(t)}}{I_{(t)}}$, в момент возникновения ступенчатого
скачка напряжения, через емкость протекает очень большой ток, ограниченный только сопротивлением
дорожек на печатной плате и сопротивлением выводов конденсаторов, которые из-за конструктивных особенностей очень
малы. А большой ток, как известно, вызывает короткое замыкание \cite{Howard J: Start Black Magic}.

Единственное отличие между схемами в наличии или отсутствии на них микросхемы LM7705 с обвязкой. Эта микросхема
предназначена для задавния на минусе питания ОУ смещения в -0,23 В, что необходимо для ухода из зоны с плохим 
напряжением смещения, см. рисунок \ref{ris:411}. Именно потребность в данной микросхеме нам и необходимо опеределить
в ходе экспериментов.

Результаты измерения без LM7705 представлены в таблице \ref{tab:I_meas_LM7705}, а с LM7705 в таблице \ref{tab:I_meas}.


\begin{table}[H]
    \centering
    \caption{Измерения тока с LM7705}
    \begin{turn}{90}
      \begin{tabular}{|c|c|c|c|c|c|c|c|c|c|c|c|c|c|}
  \cline{1-4}    \textbf{Диапазон} & \textbf{Амперный} & \textbf{Десятки мА} & \textbf{Сотни мкА} & \multicolumn{1}{c}{} & \multicolumn{1}{c}{} & \multicolumn{1}{c}{} & \multicolumn{1}{c}{} & \multicolumn{1}{c}{} & \multicolumn{1}{c}{} & \multicolumn{1}{c}{} & \multicolumn{1}{c}{} & \multicolumn{1}{c}{} & \multicolumn{1}{c}{} \bigstrut\\
  \cline{1-4}    \textbf{Uвых при 0, В} & 2,546 & 2,545 & 2,542 & \multicolumn{1}{c}{} & \multicolumn{1}{c}{} & \multicolumn{1}{c}{} & \multicolumn{1}{c}{} & \multicolumn{1}{c}{} & \multicolumn{1}{c}{} & \multicolumn{1}{c}{} & \multicolumn{1}{c}{} & \multicolumn{1}{c}{} & \multicolumn{1}{c}{} \bigstrut\\
  \cline{1-4}    \multicolumn{1}{c}{} & \multicolumn{1}{c}{} & \multicolumn{1}{c}{} & \multicolumn{1}{c}{} & \multicolumn{1}{c}{} & \multicolumn{1}{c}{} & \multicolumn{1}{c}{} & \multicolumn{1}{c}{} & \multicolumn{1}{c}{} & \multicolumn{1}{c}{} & \multicolumn{1}{c}{} & \multicolumn{1}{c}{} & \multicolumn{1}{c}{} & \multicolumn{1}{c}{} \bigstrut\\
      \hline
      \textbf{I нагрузки, мА} & -0,3  & -0,25 & -0,2  & -0,15 & -0,1  & -0,05 & 0     & 0,05  & 0,1   & 0,15  & 0,2   & 0,25  & 0,3 \bigstrut\\
      \hline
      \textbf{Uвых., В} & 1,252 & 1,506 & 1,762 & 2,017 & 2,271 & 2,527 & 2,771 & 3,038 & 3,293 & 3,548 & 3,803 & 4,058 & 4,313 \bigstrut\\
      \hline
      \multicolumn{1}{c}{} & \multicolumn{1}{c}{} & \multicolumn{1}{c}{} & \multicolumn{1}{c}{} & \multicolumn{1}{c}{} & \multicolumn{1}{c}{} & \multicolumn{1}{c}{} & \multicolumn{1}{c}{} & \multicolumn{1}{c}{} & \multicolumn{1}{c}{} & \multicolumn{1}{c}{} & \multicolumn{1}{c}{} & \multicolumn{1}{c}{} & \multicolumn{1}{c}{} \bigstrut\\
      \hline
      \textbf{I нагрузки, мА} & -30   & -25   & -20   & -15   & -10   & -5    & 0     & 5     & 10    & 15    & 20    & 25    & 30 \bigstrut\\
      \hline
      \textbf{Uвых., В} & 1,23  & 1,489 & 1,748 & 2,007 & 2,267 & 2,526 & 2,783 & 3,043 & 3,302 & 3,562 & 3,821 & 4,08  & 4,339 \bigstrut\\
      \hline
      \multicolumn{1}{c}{} & \multicolumn{1}{c}{} & \multicolumn{1}{c}{} & \multicolumn{1}{c}{} & \multicolumn{1}{c}{} & \multicolumn{1}{c}{} & \multicolumn{1}{c}{} & \multicolumn{1}{c}{} & \multicolumn{1}{c}{} & \multicolumn{1}{c}{} & \multicolumn{1}{c}{} & \multicolumn{1}{c}{} & \multicolumn{1}{c}{} & \multicolumn{1}{c}{} \bigstrut\\
      \hline
      \textbf{I нагрузки, А} & -3    & -2,5  & -2    & -1,5  & -1    & -0,5  & 0     & 0,5   & 1     & 1,5   & 2     & 2,5   & 3 \bigstrut\\
      \hline
      \textbf{Uвых., В} &       &       & 1,083 & 1,448 & 1,812 & 2,173 & 2,533 & 2,896 & 3,258 & 3,62  & 3,982 &       &  \bigstrut\\
      \hline
      \textbf{Uпадения, мВ} &       &       & -604  & -434  & -227  & -109  & 0     & 289   & 379   & 505   & 647   &       &  \bigstrut\\
      \hline
      \end{tabular}%
      \label{tab:I_meas_LM7705}%
    \end{turn}
  \end{table}

% Table generated by Excel2LaTeX from sheet 'Лист2'
\begin{table}[H]
    \centering
    \caption{Измерения тока без LM7705}
    \begin{turn}{90}
      \begin{tabular}{|c|c|c|c|c|c|c|c|c|c|c|c|c|c|}
  \cline{1-4}    \textbf{Диапазон} & \textbf{Амперный} & \textbf{Десятки мА} & \textbf{Сотни мкА} & \multicolumn{1}{c}{} & \multicolumn{1}{c}{} & \multicolumn{1}{c}{} & \multicolumn{1}{c}{} & \multicolumn{1}{c}{} & \multicolumn{1}{c}{} & \multicolumn{1}{c}{} & \multicolumn{1}{c}{} & \multicolumn{1}{c}{} & \multicolumn{1}{c}{} \bigstrut\\
  \cline{1-4}    \textbf{Uвых при 0, В} & 2,776 & 2,773 & 2,531 & \multicolumn{1}{c}{} & \multicolumn{1}{c}{} & \multicolumn{1}{c}{} & \multicolumn{1}{c}{} & \multicolumn{1}{c}{} & \multicolumn{1}{c}{} & \multicolumn{1}{c}{} & \multicolumn{1}{c}{} & \multicolumn{1}{c}{} & \multicolumn{1}{c}{} \bigstrut\\
  \cline{1-4}    \multicolumn{1}{c}{} & \multicolumn{1}{c}{} & \multicolumn{1}{c}{} & \multicolumn{1}{c}{} & \multicolumn{1}{c}{} & \multicolumn{1}{c}{} & \multicolumn{1}{c}{} & \multicolumn{1}{c}{} & \multicolumn{1}{c}{} & \multicolumn{1}{c}{} & \multicolumn{1}{c}{} & \multicolumn{1}{c}{} & \multicolumn{1}{c}{} & \multicolumn{1}{c}{} \bigstrut\\
      \hline
      \textbf{I нагрузки, мА} & -0,3  & -0,25 & -0,2  & -0,15 & -0,1  & -0,05 & 0     & 0,05  & 0,1   & 0,15  & 0,2   & 0,25  & 0,3 \bigstrut\\
      \hline
      \textbf{Uвых., В} & 1,247 & 1,502 & 1,757 & 2,011 & 2,266 & 2,521 & 2,781 & 3,033 & 3,289 & 3,543 & 3,799 & 4,053 & 4,308 \bigstrut\\
      \hline
      \multicolumn{1}{c}{} & \multicolumn{1}{c}{} & \multicolumn{1}{c}{} & \multicolumn{1}{c}{} & \multicolumn{1}{c}{} & \multicolumn{1}{c}{} & \multicolumn{1}{c}{} & \multicolumn{1}{c}{} & \multicolumn{1}{c}{} & \multicolumn{1}{c}{} & \multicolumn{1}{c}{} & \multicolumn{1}{c}{} & \multicolumn{1}{c}{} & \multicolumn{1}{c}{} \bigstrut\\
      \hline
      \textbf{I нагрузки, мА} & -30   & -25   & -20   & -15   & -10   & -5    & 0     & 5     & 10    & 15    & 20    & 25    & 30 \bigstrut\\
      \hline
      \textbf{Uвых., В} & 1,225 & 1,484 & 1,742 & 2,001 & 2,261 & 2,519 & 2,778 & 3,038 & 3,297 & 3,556 & 3,815 & 4,074 & 4,333 \bigstrut\\
      \hline
      \multicolumn{1}{c}{} & \multicolumn{1}{c}{} & \multicolumn{1}{c}{} & \multicolumn{1}{c}{} & \multicolumn{1}{c}{} & \multicolumn{1}{c}{} & \multicolumn{1}{c}{} & \multicolumn{1}{c}{} & \multicolumn{1}{c}{} & \multicolumn{1}{c}{} & \multicolumn{1}{c}{} & \multicolumn{1}{c}{} & \multicolumn{1}{c}{} & \multicolumn{1}{c}{} \bigstrut\\
      \hline
      \textbf{I нагрузки, А} & -3    & -2,5  & -2    & -1,5  & -1    & -0,5  & 0     & 0,5   & 1     & 1,5   & 2     & 2,5   & 3 \bigstrut\\
      \hline
      \textbf{Uвых., В} &       &       & 1,76  & 2,014 & 2,282 & 2,54  & 2,776 & 3,032 & 3,285 & 3,54  & 3,793 &       &  \bigstrut\\
      \hline
      \textbf{Uпадения, мВ} &       &       & -915  & -445  & -348  & -148  & 0     & 227   & 468   & 526   & 620   &       &  \bigstrut\\
      \hline
      \end{tabular}%
    \label{tab:I_meas}%
    \end{turn}
  \end{table}%
  
В качестве источника тока при проведении измерений использовался калибратор универсальный Н4-17.

Как видно из результатов измерений, дополнительное смещение отрицательного питания измерительного операционного
усилителя, обеспеченное LM7705, почти не влияет на результаты измерений. Разброс показаний между микроамперным и 
милиамперным диапазонами составляет не более 22 мВ. Показания же амперного диапазона отличаестя от остальных на 
сотни мВ, что потребует дополнительного учета при вычислении потребляемого тока. Данное различе может быть
обусловленно паразитными характеристиками платы: длинными дорожками, наводками с других линий, помехами среды. Так 
же стоит помнить по сопротивление открытого канала транзистора VT4. Для выявления более точной причины проблемы
и способа ее устранения в дальнейшем потребуется провести более подробный анализ схемы, улучшить ее топологическое
исполнение. Возможно, эту проблему получится решить программно. 

Однако, уже можно сделать вывод об избыточности LM7705 в подсистеме измерения энергопотребления.



\section{Описание схемотехнического решения}
\hspace{1cm} 

После подбора основных компонентов подсистемы измерения энергопотребления и анализа деталей схемотехнического
исполнения, можно составить схему электрическую принципиальную измерительной части отладчика, 
которая изображена на рисункем \ref{ris:431}.

\begin{figure}[H]
\centering
\includegraphics[scale = 0.6]{ris431.png}
\caption{Итоговая схема подсистемы измерения энергопотребления}
\label{ris:431}
\end{figure}

Она преимущественно повторяет решение, представленное на рисунке \ref{ris:428}, однако стоит обговорить еще раз 
принцип ее работы. С порта DUT подается потребляемый отлаживаемым устройством ток, который дальше в зависимости 
от того, какой транзистор открыт, протекает через шунты R57, R58, R59, создавая тем самым падение напряжения, 
которое попадает на неинверсный вход операционного усилителя DA2B и разница между входами ОУ усиливается в 51 
раз и подается на внутренний 12-битный АЦП микроконтроллера STM32F107, который оцифровывает значение  напряжения.

Сигнал на порты PA4\_uA\_EN, PA5\_mA\_EN, PA6\_A\_EN приходит с микроконтроллера и открывает соответствующий 
транзистор, тем самым реализуя переключение диапазонов измеряемого тока.

Конденсатор C49 является фильтрующим по питанию. 

Повторитель, реализованный на DA2A, задает смещение в 2,5 В на неинверсном входе ОУ DA2B, тем самым позволяя нам 
измерять как положительные, так и отрицательные значения токов. Так реализуется двунаправленный измеритель тока. 