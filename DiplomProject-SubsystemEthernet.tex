\chapter{Описание принципа работы подсистемы Ethernet}
\section{Разъем RJ-25, трансформатор и обвязка}
\hspace{1cm} 

Рассмотрим основные особенности, связанные с связанные со схемотехникой Ethernet части. 

На рисунке \ref{ris:RJ-45} изображена часть подсистемы от разъема RJ-45 до согласующего трансформатора. 

\begin{figure}[H]
\centering
\includegraphics[scale = 0.6]{RJ-45_Bob_Smith.png}
\caption{Часть подсистемы Ethernet от разъема RJ-45 до согласующего трансформатора}
\label{ris:RJ-45}
\end{figure}

Здесь линии TD+, TD- и RD+, RD- попарно представляют собой линии передачи и приема информации через Ethernet. 
конструктивно они реализуются в виде дифференциальных пар. 
Идея дифференциальной передачи сигналов проста. Вместо передачи одного
сигнала передаются два. Одновременно с полезным сигналом передается второй
сигнал, точно такой же, как первый сигнал, но противоположной полярности.
Возвратный ток первого сигнала — положительный. Возвратный ток второго сигнала — 
отрицательный. Они нейтрализуют друг друга \ref{ris:DiffPair}

\begin{figure}[H]
\centering
\includegraphics[scale = 0.65]{DiffPair.png}
\caption{Дифференциальная передача сигнала обеспечивает взаимную нейтрализацию
возвратных токов}
\label{ris:DiffPair}
\end{figure}

В приемнике производится сравнение двух сигналов для определения полярности логических сигналов. Для 
выполнения операции сравнения в приемнике не требуется внутреннего опорного напряжения. Напряжения сдвига земли
между передатчиком и приемником оказывают одинаковое воздействие на режим работы
обеих линий передачи, поэтому не влияют на разность сигналов, передаваемых по ним. 
Режим приема дифференциального сигнала не подвержен влиянию напряжения сдвига земли между передатчиком и 
приемником.

В случае дифференциального сигнала единственной причиной появления возвратного тока в цепи передачи является 
разбаланс сигналов дифференциальной пары. Если сигналы дифференциальной пары не идеально противоположны друг
другу, то полной взаимной нейтрализации возвратных токов сигналов не происходит. Этот ток разбаланса называется 
синфазным током. У качественного дифференциального передатчика синфазный ток в 100 раз меньше тока полезного
сигнала. Снижение синфазного тока обеспечивает снижение уровня электромагнитного излучения 
\cite{Howard J: Start Black Magic}.

Для грамотной трассировки этих линий, нужно рассчитать их ширину, с учетом конструктивных особенностей 
платы, для соблюдения необходимого импеданса, чтобы данные передавались корректно. 

Для расчета воспользуемся средствами Saturn PCB Toolkit V7.03. Согласно документации на PHY-микросхему,
целевой импеданс Ethernet линий должен быть равен 100 Ом \cite{DP83848:datasheet}. В качестве допуска 
возьмем 10\% от целевого импеданса. 

Результаты расчета представлены на рисунке \ref{ris:Saturn}

\begin{figure}[H]
\centering
\includegraphics[scale = 0.65]{Saturn.png}
\caption{Результаты расчета импеданса линий передачи данных Ethernet в Saturn Toolkit}
\label{ris:Saturn}
\end{figure}

 Весь ход расчета определяет стек платы. Мы будем использовать стандартный стек компании ООО "Резонит". 
 Плата будет двухслойная, с толщиной фольги 18 мкм, общей толщиной 1,5 мм и, следовательно, толщиной 
 диэлектрика FR4 (Tg135) 1,464 мм. 

 В разделе <<Option>> задаем конструктивные особенности платы, выбираем материал диэлектрика, его 
 диэлектрическую проницаемость, конструктивное исполнение линий. 

 Далее в <<Conductor Impedance>> задаем толщину диэлектрика и максимальную тактовую частоту платы. Теперь 
 плавно меняя значение в поле <<Conductor Width>> пытаемся добиться целевого импеданса с заданной точностью. 

 Получившиеся значения устраиваемого нас импеданса представлены на рисунке \ref{ris:Saturn}. 
 
На рисунке \ref{ris:RJ-45} конденсаторы C1-C5 и резисторы R3-R6 образуют так называемый <<Bob Smith Terminations>>. 
Их использование дает нам следующие преимущества \cite{Bob Smith terminator}:

\begin{enumerate}
    \item Уменьшение шума отражения сигнала.
    \item Улучшение времени нарастания фронтов импульса снижая уровень электромагнитных помех 
    и обеспечивая дополнительный запас по времени.
    \item Из-за уменьшения отражений мы можем увеличить длину проводимых дорожек, что облегчает трассировку.
    \item Резистивная нагрузка улучшает соотношение сигнал/шум.
\end{enumerate}


\section{PHY-контроллер}
\hspace{1cm} 


Рассмотрим рисунок \ref{ris:PhyScheme}, на котором изображено соединение между согласующим трансформатором 
и PHY-микросхемы DP83848, а так же обвязка ее и интерфейс RMII, соединяющий DP83848 и микронтроллер 
\cite{DP83848:datasheet}. 

\begin{figure}[H]
\centering
\includegraphics[scale = 0.6]{PHY scheme.png}
\caption{Результаты расчета импеданса линий передачи данных Ethernet в Saturn Toolkit}
\label{ris:PhyScheme}
\end{figure}

Данный узел схемы реализует физический уровень модели OSI. 

Резисторы R9-R12 являются согласующими для линий дифференциальной передачи, их номинал равен 50 Ом и топологически 
их следует располагать как можно ближе к PHY-микросхеме, реализуя согласование последовательным резистором на 
стороне приемника. 

Конденсаторы C6 -- C9, С12 -- С13 являются фильтрующими по питанию. 

Светодиоды VD1 и VD2 заменяют стандартные индикационные светодиоды Ethernet CPU и Link, а резисторы 
R7 и R8 задают протекающий через них ток равным примерно 10 мА.  

Выводы DA1 TX\_EN, TXD\_0, TXD\_1, RXD\_0, RXD\_1, MDC, MDO отвечают за правильную работу интерфейса
RMII, который обеспечивают передачу данных между PHY-микросхемой и микроконтроллером. 

Перечислим назначение используемых в данной схеме выводов \cite{DP83848:datasheet}.
\begin{itemize}
    \item TX\_EN -- MAC устанавливает этот сигнал, когда установлены достоверные данные на TXD.
    \item TXD\_0-1 -- передаваемые DP83848 данные.
    \item RXD\_D0-1 -- принимаемые DP83848 данные.
    \item MDC -- частота для канала MDIO.
    \item MDIO -- двунаправленный канал данных для связи с регистром STA
    \item CRS -- Опрос несущей. В течении полудуплексной операции передатчик устанавливает этот 
    вывод, когда передает или принимает пакеты данных. 
    \item RX\_ER -- флаг ошибки приема данных. 
    \item Х1 -- вход тактового сигнала.
    \item TD+/TD- -- принимаемые с Ethernet-разъема данные.
    \item RD+/RD- -- отправляемые на Ethernet-разъем данные.
\end{itemize}
 