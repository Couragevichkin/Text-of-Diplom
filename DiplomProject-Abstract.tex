\chapter*{Abstract}
\hspace{1cm}

The peculiarity of <<Internet of Things>> devices is the dependence of their power consumption 
on many external factors, which complicates their debugging directly at the workplace of the 
embedded systems programmer.
In addition, there is a need for massive remote debugging of devices using different 
radio communication protocols.
The device under development is designed to work as part of a remote access laboratory 
similar to the French FIT IoT-LAB project and implements the functionality required for 
this purpose, which is absent in the debuggers available on the market.
This document describes existing debuggers with power consumption monitoring and describes 
the process of developing such a debugger. 
In the work we analyzed the existing solutions for each functional node, as well as 
proposed a general structure of the debugger.
The work presents the implementation of the power subsystem, which consists of a PoE controller, 
an isolated fly-buck DC-DC converter and a voltage regulator.
In addition, the work describes the implementation of the physical layer of the OSI model and 
develops a bidirectional current meter based on a differential amplifier.
In addition, the paper describes the implementation of the logic voltage level conversion subsystem.
The results of the work are supposed to be used in MIEM IoT-LAB.
The work consists of 63 pages and includes 53 figures, 6 tables and 1 appendix.
The list of references includes 41 sources.

