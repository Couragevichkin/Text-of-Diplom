\begin{input}
\chapter{Введение}
\hspace{1cm} Неазвисимо от стараний разработчика или сложности проекта, большая часть времени разработки
будет потрачена на то, чтобы убедиться, что устройство работает правильно, или -- что наиболее
вероятно -- разобраться, почему устройство работает не так, как ожидалось. Отладчик -- самый мощный 
инструмент в наборе инструментов разработчика, позволяющий напрямую взаимодействовать с процессором,
задавать точки останова, пошагово управлять потоком выполнения инструкций и проверять  значения
регистров. \cite{Lakamera:embed}

Для устройств <<интернета вещей>> очень важно знать и отслеживать энергопотребление,
ведь обычно такие устройства питаются от батарейки и каждое ненужное действие уменьшит
срок службы. Мониторинг энергопотребления позволяет понять энргоэффективность каждого сенса связи,
что позволит выбрать наиболее подходящий интерфейс и протокол передачи данных.

Об актуальности возможности мониторинга энерегопотребления для отладчика говорит количество 
измерительных устройств на рынке. Характеристики основных из них приведены в таблице 1.1.

\begin{table}[H]
    \caption{Сравнение характеристик измерительных устройств}   
    \begin{center}
    \begin{tabular}{|c|c|c|c|c|}
    \hline
  Устройство & Joulescope & Otii Arc & NanoRanger & Current Ranger \\ \hline
    Диапазон тока & от -1 А до 3 А & от 0 до 2,5 А & от 1 нА до 800 мА & от -1,65 А до 3 А \\ \hline
    Разрешение & 1 нА & десятки нА & до 10 пА & до 1 пА  \\ \hline
    Погрешность & до 0,3\% & до 0,1\% & до 0,3\% & до 0,1\% \\ \hline
    Цена & 800 \$ & 700 \$ & 220 \$ & 120 \$  \\ \hline
    \end{tabular}
    \end{center}
\end{table} 

Для реализации возможности мониторинга энергопотребления устройств IoT 
в первую очередь следует начать с требований. В качестве примера можно рассматривать устройство с BLE, 
у которого с периодичностью в несколько десятков мс повторяется такой цикл: спящий режим 
с потреблением единицы мкА, далее устройство просыпается (единицы мА, время -- десятки мкс), 
передача (десятки мА, длительность передачи <<пустого>> пакета, 27 байт — около 200 мкс), 
ожидание 150 мкс inter-frame spacing, прием (единицы-десятки мА, 200 мкс). Данный паттерн поведения
и характеристики измерительных устройств из таблицы 1.1, а так же особенности IoT-устройств позволяют составить требования к разрабатываему
устройству:
\begin{itemize}
    \item полоса пропускания -- 200 кГц
    \item напряжение питания отлаживаемых устройств -- от 1,8 В до 12 В
    \item погрешность измерения -- до 0,5\%
    \item диапазон тока -- от 3,2 мА до 2 А
    \item разрешение -- 0,16 мкА
    \item время переключения диапазонов - десятки мкс
    \item себестоимость устройства -- 5000 руб.
\end{itemize}

Вышеозвученные требования позволят спроектировать устроство с учетом специфики области применения,
 правильно произвести подбор
электронной компонентной базы и сделать отладчик конкурентноспосбоным.


\end{input}