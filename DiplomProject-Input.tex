\begin{input}
\chapter{Введение}
\hspace{1cm} Неазвисимо от стараний разработчика или сложности проекта, большая часть времени разработки
будет потрачена на то, чтобы убедиться, что устройство работает правильно, или -- что наиболее
вероятно -- разобраться, почему устройство работает не так, как ожидалось. Отладчик -- самый мощный 
инструмент в наборе инструментов разработчика, позволяющий напрямую взаимодействовать с процессором,
задавать точки останова, пошагово управлять потоком выполнения инструкций и проверять  значения
регистров. 

Для устройств <<интернета вещей>> очень важно знать и отслеживать энергопотребление,
ведь обычно такие устройства питаются от батарейки и каждое ненужное действие уменьшит
срок службы. Мониторинг энергопотребления позволяет понять энргоэффективность каждого сенса связи,
что позволит выбрать более подходящий интерфейс и протокол передачи данных

Актуальность данного отладчика с мониторингом энергопотребления можно оценить, посмотря
на существующие решения.

Для реализации возможности мониторинга энергопотребления устройств IoT 
в первую очередь следует начать с требований. В качестве примера можно рассматривать устройство с BLE, 
у которого с периодичностью в несколько десятков мс повторяется такой цикл: спящий режим 
с потреблением единицы мкА, далее устройство просыпается (единицы мА, время -- десятки мкс), 
передача (десятки мА, длительность передачи <<пустого>> пакета, 27 байт — около 200 мкс), 
ожидание 150 мкс inter-frame spacing, прием (единицы-десятки мА, 200 мкс).
\end{input}