
\chapter{Описание структуры устройства}
\section{Подсистема управления}
\hspace{1cm} 

Проектирование любого устройства начинается с определения структуры, которая в дальнейшем
поможет составить его структурную схему. А главным компонентом любого устройства является
его подсистема управления.

Самые популярные подсистемы управления отладчиками базируются на микроконтроллерах,
которые поддерживает основные отладочные интерфейсы -- JTAG и SWD.
В качестве типичного <<отладочного>> микроконтроллера было решено использовать
STM32F107VCT6 из-за его следующих преимуществ:

\begin{itemize}
    \item \textit{Хорошо проработанная документация} -- компания
     STMicroelectronics является одним из лидеров на рынке микроконтроллеров, во многом благодаря
     замечательной документации, которая позволяет создавать на базе их решений проработанные
     и, по большей части, предсказуемо работающие проекты. Важно быть увереным, что при разработке
     устройства микроконтроллер не начнет показывать <<недокументированные>> возможности и
     различные баги, и репутация компании STMicroelectronics позволяет быть в этом
     уверенным. Антипримером может служить компания Espressif, чьи многочисленные ошибки,
     выявленные после выпуска очередного микроконтроллера, иногда выливаются в довольно
     объемные errata документы.
    \item \textit{Библиотеки} -- наличие удобных и, самое главное, пригодных в использовании 
     библиотек позолит значительно ускорить время разработки. STM32F107VCT6 построена на базе
     ядра Cortex-M3, для которого написано большое количество популярных библиотек, таких
     как HAL, LL, CMSIS, libopencm3 и другие.
    \item \textit{Большое количество готовых решений} -- некоторые из функций разрабатываемой
     системы могли быть реализованы ранее индивидуальным разработчиком, 
     сообществом или предприятием. Разработку всегда стоит начинать с поиска готовых или похожих 
     решений, которые, возможно, уже были разработаны и ждут интеграции в проект. Используемое
     в STM32F107VCT6 ядро сильно повышает шансы найти что-то готовое или то, что сильно 
     ускорит и упростит разработку устройства, позволяя не писать отдельные модули с <<нуля>>.
     \cite{Lakamera:embed}
    \item \textit{Доступность} -- в <<санкционную>> эпоху доступность компонента может стать 
     решающим фактором при выборе. Благодаро своей массовости микроконтроллеры серии STM32 
     можно легко найти как у дистрибьюторов ориентированных на крупные компание, так и на тех,
     кто работает с физическими лицами, что важно в рамках студентческой дипломной работы.
\end{itemize}

\section{Подсистема питания}
\hspace{1cm} 

Невозможно представить устройство без подсистемы питания, которая является его <<сердцем>>,
обеспечивая электроэнергией все остальные подсистемы. Плохо спроектированная система питания
может стать большой проблемой, вплоть до вывода из строя отдельной подсистемы или устройства
вцелом.

В качестве питания для отладчика была выбрана связка из PoE + преобразователь, выполненный по 
технологии fly-back. 

PoE (Power over Ethernet) — это технология передачи удаленным Ethernet-устройствам по 
витой паре электропитания вместе с данными. Данная технология позволяет питать подключенные 
устройства, к которым невозможно или нежелательно проводить кабели для питания

%В измерительных приборах вопрос питания стоит особенно остро, ведь даже те помехи, которые 
%не нанесли бы обычному цифровому устройству значительного вреда, могут с легкостью испортить
%всю точность измерения -----------сильная заготовка, но в другую главу-------------
